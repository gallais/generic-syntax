\subsection{Writing a Generic Scope Checker}\label{section:genericscoping}

Converting terms in the internal syntax to strings which can in turn be
displayed in a terminal or an editor window is only part of a compiler's
interaction loop. The other direction takes strings as inputs and attempts to
produce terms in the internal syntax. The first step is to parse the input
strings into structured data, the second is to perform scope checking,
and the third step consist in type checking.

Parsing is currently out of scope for our library; users can write safe
ad-hoc parsers for their object language by either using a library of total
parser combinators~\cite{DBLP:conf/icfp/Danielsson10,allais2018agdarsec}
or invoking a parser generator oracle whose target is a total
language~\cite{Stump:2016:VFP:2841316}. As we will see shortly, we can
write a generic scope checker transforming terms in a raw syntax where
variables are represented as strings into a well-scoped syntax. We will
come back to typechecking with a concrete example in section~\ref{section:typechecking}
and then discuss related future work in the conclusion.

Our scope checker will be a function taking two explicit arguments: a name for
each variable in scope \AB{Γ}, a raw term for a syntax description \AB{d} and
either fail (the Monad \AF{M} granting us the ability to fail is made explicit
in Figure~\ref{fig:scopemonad}) or return a well-scoped and sorted term for
that description.

\ExecuteMetaData[Generic/Scopecheck.tex]{totmtype}

\paragraph{Raw Terms} We can obtain \AF{Names} by reusing the standard library's
\AF{All}, a predicate transformer making sure a predicate holds of all the element
of a list. The definition of \AF{WithNames} is analogous to \AF{Pieces} in the
previous section: we expect \AF{Names} for the newly bound variables.
Terms in the raw syntax then leverage these definitions: they are either a variable
i.e. a \AD{String} potentially accompanied by some extra information \AB{E} (typically
a position in a file) or a ``layer'' of raw term where subterms are raw terms with
names for the newly-bound variables.

\begin{figure}[h]
\begin{minipage}{0.35\textwidth}
  \ExecuteMetaData[Generic/Scopecheck.tex]{names}
\end{minipage}
\begin{minipage}{0.55\textwidth}
  \ExecuteMetaData[Generic/Scopecheck.tex]{withnames}
\end{minipage}
\ExecuteMetaData[Generic/Scopecheck.tex]{raw}
\caption{Names and Raw Terms}
\end{figure}

\paragraph{Error Handling} Various things can go wrong during scope checking:
evidently a name can be out of scope but it is also possible that it may be
associated to a variable of the wrong sort. We define an enumerating type
covering these two cases. The scope checker will return a computation in the
Monad \AF{M} thus allowing us to fail and return an error, the string that
caused the failure and the extra data of type \AB{E} that accompanied it.

\begin{figure}[h]
\begin{minipage}{0.5\textwidth}
  \ExecuteMetaData[Generic/Scopecheck.tex]{error}
\end{minipage}
\begin{minipage}{0.4\textwidth}
  \ExecuteMetaData[Generic/Scopecheck.tex]{monad}
\end{minipage}
\caption{Error Type and Scope Checking Monad}\label{fig:scopemonad}
\end{figure}

Equipped with these notions, we can write down the type of \AF{toVar}
which tackles the core of the problem: variable resolution. Given a string
corresponding to a variable's name and provided that we have names for the
variables in the ambient scope and know the sort the variable needs to be
we can either fail or return a well-scoped and well-sorted \AD{Var}.

If the ambient scope is empty then we can only fail with an \AIC{OutOfScope} error.
Alternatively, if the variable's name corresponds to that of the first one
in scope we check that the sorts match up and either return \AIC{z} or fail
with a \AIC{WrongSort} error. Otherwise we look for the variable further
down the scope and use \AIC{s} to lift the result to the full scope.

\begin{figure}[h]
\ExecuteMetaData[Generic/Scopecheck.tex]{toVar}
\caption{Variable Resolution}
\end{figure}

Scope checking an entire term then amounts to lifting this action on
variables to an action on terms. The error Monad \AF{M} is by definition
an Applicative and by design our terms are
Traversable~\cite{bird_paterson_1999,DBLP:journals/jfp/GibbonsO09}.
The action on term is defined mutually with the action on scopes.
As we can see in the second equation for \AF{toScope}, thanks to the
definition of \AF{WithNames}, concrete names arrive just in time to
check the subterm with newly-bound variables.

\begin{figure}[h]
\ExecuteMetaData[Generic/Scopecheck.tex]{scopecheck}
\caption{Generic Scope Checking for Terms and Scopes}
\end{figure}
