\subsection{Binding as Self-Reference: Representing Cyclic Structures}\label{def:colist}

Ghani, Hamana, Uustalu and Vene~\citeyear{ghani2006representing} have
demonstrated how Altenkirch and Reus' type-level de Bruijn
indices~\citeyear{altenkirch1999monadic} can be used to represent
potentially cyclic structures by a finite object. In their
representation each bound variable is a pointer to the node
that introduced it. Given that we are, at the top-level, only
interested in structures with no ``dangling pointers'', we introduce
the notation \AF{TM} \AB{d} to mean closed terms (i.e. terms of type
\AD{Tm} \AB{d} \AF{∞} \AIC{[]}).

A basic example of such a structure is a potentially cyclic list which
offers a choice of two constructors: \AIC{[]} which ends the list and
\AIC{\_::\_} which combines a head and a tail but also acts as a binder
for a self-reference; these pointers can be used by using the \AIC{var}
constructor which we have renamed \AIC{↶} (pronounced ``backpointer'')
to match the domain-specific meaning.
We can see this approach in action in the examples
\AF{[0, 1]} and \AF{01↺} (pronounced ``0-1-cycle'') which describe
respectively a finite list containing
0 followed by 1 and a cyclic list starting with 0, then 1, and then
repeating the whole list again by referring to the first cons cell
represented here by the de Bruijn variable 1 (i.e. \AIC{s} \AIC{z}).

\begin{figure}[h]
\begin{minipage}{0.55\textwidth}
  \ExecuteMetaData[Generic/Examples/Colist.tex]{clistD}
  \ExecuteMetaData[Generic/Examples/Colist.tex]{clistpat}
\end{minipage}\hfill
\begin{minipage}{0.35\textwidth}
  \ExecuteMetaData[Generic/Examples/Colist.tex]{zeroones}
\end{minipage}
\caption{Potentially Cyclic Lists: Description, Pattern Synonyms and Examples}
\label{fig:examplecyclic}
\end{figure}

These finite representations are interesting in their own right
and we can use the generic semantics framework defined earlier
to manipulate them. A basic building block is the \AF{unroll}
function which takes a closed tree, exposes its top node and
unrolls any cycle which has it as its starting point. We can
decompose it using the \AF{plug} function which, given a closed
and an open term, closes the latter by plugging the former at
each free \AIC{`var} leaf. Noticing that \AF{plug}'s fundamental nature
is that of substituting a term for each leaf, it makes sense to
implement it by re-using the \AF{Substitution} semantics we already have.

\begin{figure}[h]
\begin{minipage}{0.52\textwidth}
  \ExecuteMetaData[Generic/Cofinite.tex]{plug}
\end{minipage}\hspace{2em}
\begin{minipage}{0.43\textwidth}
  \ExecuteMetaData[Generic/Cofinite.tex]{unroll}
\end{minipage}
\caption{Plug and Unroll: Exposing a Cyclic Tree's Top Layer}
\end{figure}

However, one thing still out of our reach with our current tools
is the underlying co-finite trees these finite objects are meant
to represent. We start by defining the coinductive type
corresponding to them as the greatest fixpoint of a notion of
layer. One layer of a co-finite tree is precisely given by the
meaning of its description where we completely ignore the binding
structure. We show with \AF{01⋯} the infinite list that
corresponds to the unfolding of the example \AF{01↺} given above
in Figure~\ref{fig:examplecyclic}.

%%%The definition proceeds by copattern-matching as introduced in
%%%\cite{abel2013copatterns} and showcased in \cite{thibodeau2016case}.

\begin{figure}[h]
\begin{minipage}{0.5\textwidth}
  \ExecuteMetaData[Generic/Cofinite.tex]{cotm}
\end{minipage}\hfill
\begin{minipage}{0.4\textwidth}
  \ExecuteMetaData[Generic/Examples/Colist.tex]{zeroones2}
\end{minipage}
\caption{Co-finite Trees: Definition and Example}
\end{figure}

We can then make the connection between potentially cyclic
structures and the co-finite trees formal by giving an \AF{unfold}
function which, given a closed term, produces its unfolding.
The definition proceeds by unrolling the term's top layer and
co-recursively unfolding all the subterms.

\begin{figure}[h]
 \ExecuteMetaData[Generic/Cofinite.tex]{unfold}
\caption{Generic Unfold of Potentially Cyclic Structures}
\end{figure}

Even if the
powerful notion of semantics described in Section~\ref{section:semantics}
cannot encompass all the traversals we may be interested in,
it provides us with reusable building blocks: the definition
of \AF{unfold} was made very simple by reusing the generic
program \AF{fmap} and the \AF{Substitution} semantics whilst
the definition of \AR{∞Tm} was made easy by reusing \AF{⟦\_⟧}.
