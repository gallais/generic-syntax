\subsection{Generic Decidable Equality for Terms}

Haskell programmers are used to receiving help from the `deriving'
mechanism~\cite{DBLP:journals/entcs/HinzeJ00,DBLP:conf/haskell/MagalhaesDJL10}
to automatically generate common traversals for every inductive type
they define. Recalling that generic programming is normal programming
over a universe in a dependently typed
language~\cite{DBLP:conf/ifip2-1/AltenkirchM02}, we ought to be able to
deliver similar functionalities.

We will focus in this section on the definition of an equality test. The
techniques used in this concrete example are general enough that they also
apply to the definition of an ordering test, a \texttt{Show} instance, etc.
In type theory we can do better than an uninformative boolean function
claiming that two terms are equal: we can implement a decision procedure
for propositional equality~\cite{DBLP:conf/icfp/LohM11} which either
returns a proof that its two inputs are equal or a proof that they
cannot possibly be.

The notion of decidability can be neatly formalised by an inductive family
with two constructors: a \AF{Set} \AB{P} is decidable if we can either say
\AIC{yes} and return a proof of \AB{P} or \AIC{no} and provide a proof of
the negation of \AB{P} (here, a proof that \AB{P} implies the empty type
\AD{⊥}).

\begin{figure}[h]
\begin{minipage}[t]{0.45\textwidth}
  \ExecuteMetaData[Stdlib.tex]{bottom}
\end{minipage}
\begin{minipage}[t]{0.45\textwidth}
  \ExecuteMetaData[Stdlib.tex]{dec}
\end{minipage}
\caption{Empty Type and Decidability as an Inductive Family}
\end{figure}

To get acquainted with these new notions we can start by proving that equality
of variables is decidable.

\subsubsection{Deciding Variable Equality}

The type of the decision procedure for equality of variables is as follows:
given any two variables (of the same type, in the same context), the set of
equality proofs between them is \AD{Dec}idable.

\ExecuteMetaData[Generic/Equality.tex]{eqVarType}

We can easily dismiss two trivial cases: if the two variables have distinct
head constructors then they cannot possibly be equal. Agda allows us to
dismiss the impossible premisse of the function stored in the \AIC{no}
contructor by using an absurd pattern \AS{()}.

\ExecuteMetaData[Generic/Equality.tex]{eqVarNo}

Otherwise if the two head constructors agree we can be in one of two
situations. If they are both \AIC{z} then we can conclude that the two
variables are indeed equal to each other.

\ExecuteMetaData[Generic/Equality.tex]{eqVarYesZ}

Finally if the two variables are {(\AIC{s} \AB{v})} and {(\AIC{s} \AB{w})}
respectively then we need to check recursively whether \AB{v} is equal
to \AB{w}. If it is the case we can conclude by invoking the congruence
rule for \AIC{c}. If \AB{v} and \AB{w} are not equal then a proof that
{(\AIC{s} \AB{v})} and {(\AIC{s} \AB{w})} are will lead to a direct
contradiction by injectivity of the constructor \AIC{s}.

\ExecuteMetaData[Generic/Equality.tex]{eqVarYesS}

\subsubsection{Deciding Term Equality}

The constructor \AIC{`σ} for descriptions gives us the ability to store
values of any \AF{Set} in terms. For some of these \AF{Set}s (e.g.
{(\AD{ℕ} → \AD{ℕ})}), equality is not decidable. As a consequence
our decision procedure will be conditioned to the satisfaction of a
certain set of \AF{Constraints} which we can compute from the \AD{Descr}
itself, as show in Figure~\ref{fig:eqconstraints}. We demand that we are
able to decide equality for all of the \AF{Set}s mentioned in a description.

\begin{figure}[h]
\ExecuteMetaData[Generic/Equality.tex]{constraints}
\caption{Constraints Necessary for Decidable Equality}\label{fig:eqconstraints}
\end{figure}

Remembering that our descriptions are given a semantics as a big right-nested
product terminated by an equality constraint, we realise that proving decidable
equality will entail proving equality between proofs of equality. We are happy
to assume Streicher's axiom K~\cite{DBLP:conf/lics/HofmannS94} to easily
dismiss this case. A more conservative approach would be to demand that equality
is decidable on the index type \AB{I} and to then use the classic Hedberg
construction~\cite{DBLP:journals/jfp/Hedberg98} to recover uniqueness of
identity proofs for \AB{I}.

Assuming that the constraints computed by {(\AF{Constraints} \AB{d})} are
satisfied, we define the decision procedure for equality of terms together
with its equivalent for bodies. The function \AF{eq\textasciicircum{}Tm}
is a straightforward case analysis dismissing trivially impossible cases
where terms have distinct head constructors (\AIC{`var} vs. \AIC{`con})
and using either \AF{eq\textasciicircum{}Var} or \AF{eq\textasciicircum{}⟦⟧}
otherwise. The latter is defined by induction over \AB{e}. The somewhat
verbose definitions are not enlightening so we leave them out here.


\begin{figure}[h]
\ExecuteMetaData[Generic/Equality.tex]{eqTmType}
\caption{Type of Decidable Equality for Terms and Bodies}\label{fig:eqtype}
\end{figure}

We now have an informative decision procedure for equality between terms
provided that the syntax they belong to satisfies a set of constraints.
Other generic generic functions and decision procedure can be defined
following the same approach: implement a similar function for variables
first, compute a set of constraints, and demonstrate that they are
sufficient to handle any input term.
