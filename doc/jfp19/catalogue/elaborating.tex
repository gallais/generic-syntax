\subsection{An Algebraic Approach to Elaboration}\label{section:elaboration}

Instead of generating a type or checking that a candidate will do, we can use
our language of \AD{Desc}riptions to describe not only the source language but
also a language of evidence. During typechecking we generate at the same time
an expression's type and a well scoped and well typed term of that type. We use
\AF{STLC} (defined in Section~\ref{par:intrinsicSTLC}) as our \emph{internal}
language.

Before we can jump right in, we need to set the stage: a \AR{Semantics} for a
\AF{Bidi} term will involve ({\AD{Mode} \AF{─Scoped}}) notions of values and
computations but an \AF{STLC} term is ({\AD{Type} \AF{─Scoped}}). We first
introduce a \AF{Typing} associating types to each of the modes in scope, together
with \AF{fromTyping} extracting the context thus defined.

\begin{figure}[h]
\begin{minipage}{0.4\textwidth}
  \ExecuteMetaData[Generic/Semantics/Elaboration/Typed.tex]{typing}
\end{minipage}
\begin{minipage}{0.5\textwidth}
  \ExecuteMetaData[Generic/Semantics/Elaboration/Typed.tex]{fromtyping}
\end{minipage}
\caption{Typing: From Contexts of \AD{Mode}s to Contexts of \AD{Type}s\label{fig:typingmodes}}
\end{figure}

We can then explain what it means for elaboration to target \AB{T} a
({\AD{Type} \AF{─Scoped}}) at a type \AB{σ}: provided a list of modes and a corresponding
typing, we should get a \AB{T} of type \AB{σ} in the context induced by that \AF{Typing}.

\begin{figure}[h]
\ExecuteMetaData[Generic/Semantics/Elaboration/Typed.tex]{elab}
\caption{Elaboration of a Scoped Family}
\end{figure}

In particular, our environment values are elaboration functions targeting \AD{Var}. We
expect all values to be in scope i.e. provided any typing of the scope of modes, we are
guaranteed to return a type together with a variable of that type in the context induced
by the typing. We once more limit environment values to the \AIC{Infer} mode only.

\begin{figure}[h]
\ExecuteMetaData[Generic/Semantics/Elaboration/Typed.tex]{varmode}
\caption{Values as Variables and Inference Functions\label{fig:elabvalues}}
\end{figure}

The computations are a bit more tricky. On the one hand, if we are in checking mode
then we expect that for any typing of the scope of modes and any type candidate we
can \AD{Maybe} return a term at that type in the induced context. On the other hand,
in the inference mode we expect that given any typing of the scope, we can \AD{Maybe}
return a type together with a term at that type in the induced context.

\begin{figure}[h]
\ExecuteMetaData[Generic/Semantics/Elaboration/Typed.tex]{elabmode}
\caption{Computations as \AD{Mode}-indexed Elaboration Functions\label{fig:elabcomputations}}
\end{figure}

Because we are now writing a typechecker which returns evidence of its claims, we need
more informative variants of the equality and \AF{isArrow} checks. In the equality
checking case we want to get a proof of propositional equality but we only care
about the successful path and will happily return \AIC{nothing} when failing.
Agda's support for (dependent!) \AK{do}-notations makes writing the check
really easy. For the arrow type, we introduce a family \AD{Arrow} constraining the
shape of its index to be an arrow type and redefine \AF{isArrow} as a view targeting
this inductive family~\cite{DBLP:conf/popl/Wadler87}.

\begin{figure}[h]
\begin{minipage}{0.45\textwidth}
  \ExecuteMetaData[Generic/Semantics/Elaboration/Typed.tex]{equal}
\end{minipage}
\begin{minipage}{0.45\textwidth}
  \ExecuteMetaData[Generic/Semantics/Elaboration/Typed.tex]{arrow}
\end{minipage}
\caption{Informative Equality Check and Arrow View\label{fig:informativecheck}}
\end{figure}

We now have all the basic pieces and can start writing elaboration code. We
once more start by dealing with each constructor in isolation before putting
everything together to get a \AR{Semantics}. These steps are very similar to
the ones in the previous section.

\paragraph{Application} In the application case, we start by elaborating the
function and we get its type together with its internal representation. We then
check that the inferred type is indeed an \AD{Arrow} and elaborate the argument
using the corresponding domain. We conclude by returning the codomain together
with the internal function applied to the internal argument.
\ExecuteMetaData[Generic/Semantics/Elaboration/Typed.tex]{app}
\paragraph{λ-abstraction} For the λ-abstraction case, we start by checking that
the type candidate is an \AD{Arrow}. We can then elaborate the body of the lambda
in a context extended with one \AIC{Infer} variable assigned an inference function
thanks to the auxiliary function \AF{var0}. From this we get an internal term
corresponding to the body of the λ-abstraction and conclude by returning it wrapped
in a \AIC{`lam} constructor.
\ExecuteMetaData[Generic/Semantics/Elaboration/Typed.tex]{lam}
\paragraph{Cut: A \AIC{Check} in an \AIC{Infer} position} For cut, we start by
elaborating the term with the type annotation provided and return them paired
together.
\ExecuteMetaData[Generic/Semantics/Elaboration/Typed.tex]{cut}
\paragraph{Embedding of \AIC{Infer} into \AIC{Check}} For the change of direction
\AIC{Emb} we not only want to check that the inferred type and the type candidate
are equal: we need to cast the internal term labelled with the inferred type to
match the type candidate. Luckily, Agda's dependent \AK{do}-notations make once
again our job easy: when we make the pattern \AIC{refl} explicit, the equality holds
in the rest of the block.
\ExecuteMetaData[Generic/Semantics/Elaboration/Typed.tex]{emb}

We have almost everything we need to define elaboration as a semantics. Discharging
the \ARF{th\textasciicircum{}𝓥} constraint is a bit laborious and the proof doesn't
yield any additional insight so we leave it out here. The semantical counterpart of
variables (\ARF{var}) is fairly straightforward: provided a \AF{Typing}, we run the
inference and touch it up to return a term rather than a mere variable. Finally we
define the algebra (\ARF{alg}) by pattern-matching on the constructor and using our
previous combinators.

We can once more define a specialised version of the traversal induced by this
\AR{Semantics} for closed terms: not only can we give a (trivial) initial
environment but we can also give a (trivial) initial \AF{Typing}. This leads to
the following definitions:

\begin{figure}[h]
\ExecuteMetaData[Generic/Semantics/Elaboration/Typed.tex]{typemode}
\ExecuteMetaData[Generic/Semantics/Elaboration/Typed.tex]{type-}
\caption{Evidence-producing Type (Checking / Inference) Function}
\end{figure}

Revisiting the example introduced in Section~\ref{section:typechecking},
we can check that elaborating the expression {(λx.x : (α→α)→(α→α))(λx.x)}
yields the type {(α→α)} together with the term {(λx.x)(λx.x)} in internal
syntax. Type annotations have disappeared as all the type invariants are
enforced intrinsically.

\begin{figure}[h]
\begin{minipage}{0.35\textwidth}
  \ExecuteMetaData[Generic/Semantics/Elaboration/Typed.tex]{identities}
\end{minipage}
\begin{minipage}{0.55\textwidth}
  \ExecuteMetaData[Generic/Semantics/Elaboration/Typed.tex]{example}
\end{minipage}
\end{figure}
