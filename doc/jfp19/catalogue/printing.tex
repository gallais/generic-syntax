
\subsection{Printing with Names}\label{section:genericprinting}

We have seen in Section~\ref{section:printing} that printing with names
is an instance of ACMM's notion of \semrec{}. We will now show that this
observation can be generalised to arbitrary syntaxes with binding. Unlike
renaming or substitution, this generic program will require user guidance:
there is no way for us to guess how an encoded term should be printed. We
can however take care of the name generation (using the \AF{Fresh} monad from Page~\pageref{section:printing}), deal with variable binding,
and implement the traversal generically. We want our printer to have type:
\begin{agdasnippet}
\ExecuteMetaData[Generic/Semantics/Printing.tex]{printtype}
\end{agdasnippet}
%
where \AF{Display} explains how to print one `layer' of term provided that
we are handed the \AF{Pieces} corresponding to the printed subterm and
names for the bound variables:
\begin{agdasnippet}
  \ExecuteMetaData[Generic/Semantics/Printing.tex]{display}
\end{agdasnippet}
%
Reusing the notion of \AR{Name} introduced in Section~\ref{section:printing},
we can make \AF{Pieces} formal. A subterm has already been printed if we
have a string representation of it together with an environment of \AR{Name}s
we have attached to the newly-bound variables this structure contains.
That is to say:
%
\begin{agdasnippet}
\ExecuteMetaData[Generic/Semantics/Printing.tex]{pieces}
\end{agdasnippet}
%
The key observation that will help us define a generic printer is that
\AF{Fresh} composed with \AR{Name} is \AR{VarLike}. Indeed, as the composition
of a functor and a trivially thinnable \AR{Wrap}per, \AF{Fresh} is \AF{Thinnable},
and \AF{fresh} (defined on Page~\pageref{section:printing}) is the proof that we
can generate placeholder values thanks to the name supply.

\begin{agdasnippet}
\ExecuteMetaData[Generic/Semantics/Printing.tex]{vlmname}
\end{agdasnippet}

This \AR{VarLike} instance empowers us to reify in an effectful manner
a \AF{Kripke} function space taking \AF{Name}s and returning a \AF{Printer}
to a set of \AF{Pieces}.

\begin{agdasnippet}
\ExecuteMetaData[Generic/Semantics/Printing.tex]{reifytype}
\end{agdasnippet}

In case there are no newly bound variables, the \AF{Kripke} function space
collapses to a mere \AR{Printer} which is precisely the wrapped version of
the type we expect.

\begin{agdasnippet}
\ExecuteMetaData[Generic/Semantics/Printing.tex]{reifybase}
\end{agdasnippet}

Otherwise we proceed in a manner reminiscent of the pure reification function
defined at the end of Section~\ref{section:renandsub}. We start by generating an environment
of names for the newly-bound variables by using the fact that \AF{Fresh} composed
with \AF{Name} is \AR{VarLike} together with the fact that environments are
Traversable~\cite{mcbride_paterson_2008}, %%%
and thus admit the standard Haskell-like \AF{mapA} and \AF{sequenceA}
traversals. %%%
We then run the \AF{Kripke} function
on these names to obtain the string representation of the subterm. We finally
return the names we used together with this string.

\begin{agdasnippet}
\ExecuteMetaData[Generic/Semantics/Printing.tex]{reifypieces}
\end{agdasnippet}

We can put all of these pieces together to obtain the \AF{Printing} semantics
presented in Figure~\ref{fig:genericprinting}.
The first two constraints can be trivially discharged. When defining the
algebra we start by reifying the subterms, then use the fact that  one ``layer''
of term of our syntaxes with binding is always traversable to combine all of
these results into a value we can apply our display function to.

\begin{figure}[h]
\ExecuteMetaData[Generic/Semantics/Printing.tex]{printing}
\caption{Printing with \AF{Name}s as a \AR{Semantics}}\label{fig:genericprinting}
\end{figure}

This allows us to write a \AF{printer} for open terms as demonstrated in
Figure~\ref{fig:genericprint}.
\reva{We start by using \AF{base} (defined in Section~\ref{sec:varlike:base})
}
to generate an environment of \AR{Name}s for the free variables, then use
our semantics to get a \AF{printer} which we can run using a stream \AF{names} of distinct
strings as our name supply.

\begin{figure}[h]
\ExecuteMetaData[Generic/Semantics/Printing.tex]{print}
\caption{Generic Printer for Open Terms}\label{fig:genericprint}
\end{figure}


\paragraph{Untyped λ-calculus} Defining a printer for the untyped
λ-calculus is now very easy: we define a \AF{Display} by case analysis.
In the application case, we combine the string representation of the
function, wrap its argument's representation between parentheses and
concatenate the two together. In the lambda abstraction case, we are
handed the name the bound variable was assigned together with the body's
representation; it is once more a matter of putting the \AF{Pieces}
together.

\begin{agdasnippet}
\ExecuteMetaData[Generic/Examples/Printing.tex]{printUTLC}
\end{agdasnippet}

As always, these functions are readily executable and we can check
their behaviour by writing tests. First, we print the identity function
defined in Section~\ref{section:universe}
in an empty context and verify that we do obtain the string \AStr{"λa. a"}.
Next, we print an open term in a context of size two and can immediately
observe that names are generated for the free variables first, and then the
expression itself is printed.

\begin{minipage}[t]{0.45\textwidth}
  \begin{agdasnippet}
  \ExecuteMetaData[Generic/Examples/Printing.tex]{printid}
  \end{agdasnippet}
\end{minipage}
\begin{minipage}[t]{0.45\textwidth}
  \begin{agdasnippet}
  \ExecuteMetaData[Generic/Examples/Printing.tex]{printopen}
  \end{agdasnippet}
\end{minipage}
