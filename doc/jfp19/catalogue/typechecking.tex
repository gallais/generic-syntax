\subsection{An Algebraic Approach to Typechecking}\label{section:typechecking}

Following Atkey~\citeyear{atkey2015algebraic}, we can consider type checking
and type inference as a possible semantics for a bi-directional~\cite{pierce2000local}
language. We reuse the syntax introduced in Section~\ref{par:bidirectional}; it
gives us a simply typed bi-directional calculus as a bi-sorted language using
a notion of \AD{Mode} to distinguish between terms for which we will be able to
\AIC{Infer} the type and the ones for which we will have to \AIC{Check} a type
candidate.

The values stored in the environment of the typechecking function attach \AD{Type}
information to bound variables whose \AD{Mode} is \AIC{Infer}, guaranteeing no
variable ever uses the \AIC{Check} mode. In contrast, the generated computations
will, depending on the mode, either take a type candidate and \AIC{Check} it is
valid or \AIC{Infer} a type for their argument. These computations are always
potentially failing so we use the \AD{Maybe} monad.

\begin{figure}[h]
\begin{minipage}{0.40\textwidth}
  \ExecuteMetaData[Generic/Semantics/TypeChecking.tex]{varmode}
\end{minipage}\hfill
\begin{minipage}{0.50\textwidth}
  \ExecuteMetaData[Generic/Semantics/TypeChecking.tex]{typemode}
\end{minipage}
\caption{Var- and Type- Relations indexed by the Mode}
\end{figure}

A change of direction from inferring to checking will require being able to check
that two types agree so we introduce the function \AF{\_=?\_}. Similarly we will
sometimes expect a function type but may be handed anything so we will have to check
with \AF{isArrow} that our candidate's head constructor is indeed an arrow, and
collect the domain and codomain.

\begin{figure}[h]
\begin{minipage}{0.45\textwidth}
  \ExecuteMetaData[Generic/Semantics/TypeChecking.tex]{typeeq}
\end{minipage}\hfill
\begin{minipage}{0.45\textwidth}
  \ExecuteMetaData[Generic/Semantics/TypeChecking.tex]{isArrow}
\end{minipage}
\caption{Tests for \AD{Type} values}
\end{figure}

We can now define typechecking as a \semrec{}. We describe the algorithm constructor
by constructor; in the \AR{Semantics} definition (omitted here) the algebra will
simply perform a dispatch and pick the relevant auxiliary lemma. Note that in the
following code, \AF{\_<\$\_} is, following classic Haskell notations, the function
which takes an \AB{A} and a {\AD{Maybe} \AB{B}} and returns a {\AD{Maybe} \AB{A}}
which has the same structure as its second argument.

\paragraph{Application} When facing an application: infer the type of the function,
make sure it is an arrow type, check the argument at the domain's type and return
the codomain.
\ExecuteMetaData[Generic/Semantics/TypeChecking.tex]{app}
\paragraph{λ-abstraction} For a λ-abstraction: check the input
type is an arrow type and check the body at the codomain type in the extended
environment where the newly-bound variable is of mode \AIC{Infer} and has the
domain's type.
\ExecuteMetaData[Generic/Semantics/TypeChecking.tex]{lam}
\paragraph{Cut: A \AIC{Check} in an \AIC{Infer} position} A cut always comes
with a type candidate against which to check the term and to be returned in
case of success.
\ExecuteMetaData[Generic/Semantics/TypeChecking.tex]{cut}
\paragraph{Embedding of \AIC{Infer} into \AIC{Check}} Finally, the change of
direction from \AIC{Infer}rable to \AIC{Check}able is successful when the
inferred type is equal to the expected one.
\ExecuteMetaData[Generic/Semantics/TypeChecking.tex]{emb}

We have defined a bidirectional typechecker for this simple language by
leveraging the \semrec{} framework. We can readily run it on closed terms
using the \AF{closed} corollary defined in Figure~\ref{fig:evalclosed}
and (defining \AB{β} to be {(\AB{α} \AIC{`→} \AB{α})}) infer the type of
the expression {(λx. x : β → β) (λx. x)}.

\begin{figure}[h!]
\begin{minipage}{0.35\textwidth}
  \ExecuteMetaData[Generic/Semantics/TypeChecking.tex]{type-}
\end{minipage}\hfill
\begin{minipage}{0.55\textwidth}
  \ExecuteMetaData[Generic/Semantics/TypeChecking.tex]{example}
\end{minipage}
\caption{Type- Inference / Checking as a Semantics}
\end{figure}

The output of this function is not very informative. As we will see shortly,
there is nothing stopping us from moving away from a simple computation
returning a {(\AD{Maybe} \AD{Type})} to an evidence-producing function
elaborating a term in \AF{Bidi} to a well scoped and typed term in \AF{STLC}.
