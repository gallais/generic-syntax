\subsection{Sugar and desugaring as a semantics}\label{section:letbinding}

One of the advantages of having a universe of programming language
descriptions is the ability to concisely define an \emph{extension}
of an existing language by using \AD{Desc}ription transformers
grafting extra constructors à la Swiestra~\citeyear{swierstra_2008}.
This is made extremely simple by the disjoint sum combinator
\AF{\_`+\_} which we defined in Section~\ref{section:universe}.
An example of such an extension is the addition of let-bindings to
an existing language.

Let bindings allow the user to avoid repeating themselves by naming
sub-expressions and then using these names to refer to the associated
terms. Preprocessors adding these types of mechanisms to existing
languages (from C to CSS) are rather popular. We introduce a
description \AD{Let} which can be used to extend any language
description \AB{d} to a language with let-bindings (\AB{d} \AF{`+}
\AF{Let}).

\begin{minipage}{\textwidth}
\begin{minipage}[t]{0.45\textwidth}
  \ExecuteMetaData[Generic/Syntax/LetBinder.tex]{letcode}
\end{minipage}
\begin{minipage}[t]{0.45\textwidth}
  \ExecuteMetaData[Generic/Syntax/LetBinder.tex]{letpattern}
\end{minipage}
% \caption{Description of a single let binding, associated pattern synonyms
%   \label{defn:letD}}
\end{minipage}

This description states that a let-binding node stores a pair of types
\AB{$\sigma$} and \AB{$\tau$} and two subterms. First comes the let-bound
expression of type \AB{$\sigma$} and second comes the body of the let which
has type \AB{$\tau$} in a context extended with a fresh variable of type
\AB{$\sigma$}. This defines a term of type \AB{$\tau$}.

In a dependently typed language, a type may depend on a value which
in the presence of let bindings may be a variable standing for an
expression. The user naturally does not want it to make any difference
whether they used a variable referring to a let-bound expression or
the expression itself. Various type checking strategies can accommodate
this expectation: in Coq~\cite{Coq:manual} let bindings are primitive
constructs of the language and have their own typing and reduction
rules whereas in Agda they are elaborated away to the core language
by inlining.

This latter approach to extending a language \AB{d} with let bindings
by inlining them before type checking can be implemented generically as
a semantics over (\AB{d} \AF{`+} \AF{Let}). For this semantics values
in the environment and computations are both let-free terms. The algebra
of the semantics can be defined by parts thanks to \AF{case}, the eliminator
for \AF{\_`+\_} defined in Section~\ref{section:universe}:
the old constructors are kept the same by
interpreting them using the generic substitution algebra (\AF{Sub});
whilst the let-binder precisely provides the extra value to be added to the
environment.

\begin{agdasnippet}
  \ExecuteMetaData[Generic/Semantics/Elaboration/LetBinder.tex]{unletcode}
%\caption{Desugaring as a \AR{Semantics}\label{defn:UnLet}}
\end{agdasnippet}

The process of removing let binders is then kickstarted with the placeholder
environment \AF{id\textasciicircum{}Tm}~=~\AIC{pack}~\AIC{`var} 
of type {(\AB{Γ} \AR{─Env}) (\AD{Tm} \AB{d} ∞) \AB{Γ}}. 

\begin{agdasnippet}
  \ExecuteMetaData[Generic/Semantics/Elaboration/LetBinder.tex]{unlet}
%\caption{Specialising \AF{semantics} with an environment of placeholder values\label{defn:unlet}}
\end{agdasnippet}

In less than 10 lines of code we have defined a generic extension of
syntaxes with binding together with a semantics which corresponds
to an elaborator translating away this new construct.
\revb{In previous work~\cite{allais2017type}, we focused on STLC only
and showed that it is similarly possible to implement a Continuation
Passing Style transformation as the composition of two semantics
à la Hatcliff and Danvy~\citeyear{hatcliff1994generic}.
The first semantics embeds STLC into Moggi's
Meta-Language~\citeyear{DBLP:journals/iandc/Moggi91} and thus fixes
an evaluation order. The second one translates Moggi's ML back into
STLC in terms of explicit continuations with a fixed return type.
}

We have demonstrated how easily one can define extensions and combine
them on top of a base language without having to reimplement common
traversals for each one of the intermediate representations. Moreover,
it is possible to define \emph{generic} transformations elaborating
these added features in terms of lower-level ones. This suggests that
this setup could be a good candidate to implement generic compilation
passes and could deal with a framework using a wealth of slightly
different intermediate languages à la Nanopass~\cite{Keep:2013:NFC:2544174.2500618}.
