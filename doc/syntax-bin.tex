
\subsection{Binding as Self-Reference: Representing Cyclic Structures}

Ghani, Hamana, Uustalu and Vene~\cite{ghani2006representing} have
demonstrated how Altenkirch and Reus' type-level de Bruijn
indices~\cite{altenkirch1999monadic} can be used to represent
potentially cyclic structures by a finite object. In their
representation each bound variable is a pointer to the node
that introduced it. Because we are, at the top-level, only
interested in structures with no ``dangling pointers'', we introduce
the notation \AF{TM} \AB{d} to mean closed terms (i.e. terms of type
\AD{Tm} \AB{d} \AF{∞} \AN{0}).

A basic example of such a structure is a potentially cyclic list which
offers a choice of two constructors: nil which ends the list and cons
which combines a head and a tail but also acts as a binder for a
self-reference. We can see this approach in action in the example \AF{01↺}
which describes a cyclic list starting with 0, then 1, and then
repeats the whole list again by referring to the first cons cell
represented here by the variable \AIC{suc} \AIC{zero}.

\begin{figure}[h]
\begin{minipage}{0.45\textwidth}
  \ExecuteMetaData[generic-cofinite.tex]{clistD}
\end{minipage}\hspace{2em}
\begin{minipage}{0.45\textwidth}
  \ExecuteMetaData[generic-cofinite.tex]{zeroones}
\end{minipage}
\caption{Potentially Cyclic Lists: Description and Example}
\end{figure}

These finite representations are interesting in their own right
and we can use the generic semantics framework defined earlier
to manipulate them. A basic building block is the \AF{unroll}
function which takes a closed tree, exposes its top node and
unrolls any cycle which has it as its starting point. We can
decompose it using the \AF{plug} function which, given a closed
and an open term, closes the latter by plugging the former at
each \AIC{`var} leaf. Because \AF{plug}'s fundamental nature
is that of substituting a term for each leaf, it can naturally
be implemented using \AF{Substitution}.

\begin{figure}[h]
\begin{minipage}{0.52\textwidth}
  \ExecuteMetaData[generic-cofinite.tex]{plug}
\end{minipage}\hspace{2em}
\begin{minipage}{0.43\textwidth}
  \ExecuteMetaData[generic-cofinite.tex]{unroll}
\end{minipage}
\caption{Plug and Unroll: Exposing a Cyclic Tree's Top Layer}
\end{figure}

However, one thing still out of our reach with the current tools we have
is the underlying co-finite trees these finite objects are meant
to represent. We can start by defining the coinductive type
corresponding to them as the greatest fixpoint of a notion of
layer. One layer of a co-finite tree is precisely given by the
meaning of its description where we completely ignore the binding
structure. We show with \AF{01⋯} which infinite list ought to
correspond to the cyclic example \AF{01↺} given above.

\begin{figure}[h]
\begin{minipage}{0.55\textwidth}
  \ExecuteMetaData[generic-cofinite.tex]{cotm}
\end{minipage}\hspace{2em}
\begin{minipage}{0.35\textwidth}
  \ExecuteMetaData[generic-cofinite.tex]{zeroones2}
\end{minipage}
\caption{Co-finite Trees: Definition and Example}
\end{figure}

We can then make the connection between potentially cyclic
structures and the co-finite trees by giving an \AF{unfold}
function which, given a closed term, produces its unfolding.
The definition proceeds by unrolling the term's top layer and
co-recursively unfolding all the subterms.

\begin{figure}[h]
 \ExecuteMetaData[generic-cofinite.tex]{unfold}
\caption{Generic Unfold of Potentially Cyclic Structures}
\end{figure}

We can see that this universe of descriptions allows us to
implement generic functions once and for all. Even if the
powerful notion of semantics described in Section~\ref{section:semantics}
cannot encompass all the traversals we may interested in,
it provides us with reusable building blocks: the definition
of \AF{unfold} was made very simple by reusing the generic
program \AF{fmap} and the \AF{Substitution} semantics.












\subsection{Definition of Bisimilarity for co-finite objects}

Although we were able to use propositional equality when studying
syntactic traversals working on terms, it is not the appropriate
notion of equality for co-finite trees. We can use \AF{Zip} to
define generically a coinductive notion of bisimilarity for all
co-finite trees obtained as unfolding of descriptions. And we
can derive from \AF{Zip}'s properties the fact that this gives
rise to an equivalence relation.

\begin{figure}[h]
 \ExecuteMetaData[generic-bisimilar.tex]{bisim}
 \ExecuteMetaData[generic-bisimilar.tex]{eqrel}
\caption{Bisimilarity, an Equivalence Relation for Co-finite Trees}
\end{figure}





We have also seen how this setup
could be applied to a different domain: the representation of potentially
cyclic structures by finite artefacts.


The potentially cyclic structures we have studied have been improved
upon by Hamana~\citeyear{Hamana2009} who gave a presentation which
preserves sharing: pointers can not only refer to nodes above them
but also across from them in the cyclic tree. This yields a much more
involved binding structure which would be interesting in its own right.
